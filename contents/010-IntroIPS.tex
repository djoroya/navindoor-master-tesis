\chapter{Sistemas de posicionamiento en interiores}

% -
\section{Clasificación de IPS}


\subsection{Triangulación mediante RSS y ToF}

% Yang, Jie; Chen, Yingying (2009-11-01). Indoor Localization Using Improved RSS-Based Lateration Methods. IEEE Global Telecommunications Conference, 2009. GLOBECOM 2009. pp. 1–6. CiteSeerX 10.1.1.386.4258. doi:10.1109/GLOCOM.2009.5425237. ISBN 978-1-4244-4148-8.

% Kotaru, Manikanta; Joshi, Kiran; Bharadia, Dinesh; Katti, Sachin (2015-01-01). SpotFi: Decimeter Level Localization Using WiFi. Proceedings of the 2015 ACM Conference on Special Interest Group on Data Communication. SIGCOMM '15. New York, NY, USA: ACM. pp. 269–282. 
\subsection{Trilateración mediante AoA}

\subsection{Fingerprinting}

% [1] P. Bahl and V. N. Padmanabhan, “RADAR: an in-building RF-based user location and tracking system,” in Proceedings of 19th Annual Joint Conference of the IEEE Computer and Communications Societies (INFOCOM ’00), vol. 2, pp. 775–784, Tel Aviv.Israel, March 2000.

% [3] Youssef, M. A.; Agrawala, A.; Shankar, A. Udaya (2003-03-01). WLAN location determination via clustering and probability distributions. Proceedings of the First IEEE International Conference on Pervasive Computing and Communications, 2003. (PerCom 2003). pp. 143–150. CiteSeerX 10.1.1.13.4478. doi:10.1109/PERCOM.2003.1192736. ISBN 978-0-7695-1893-0.

La toma de huellas dactilares tradicional también se basa en RSSI, pero simplemente se basa en el registro de la intensidad de la señal desde varios puntos de acceso en el rango y el almacenamiento de esta información en una base de datos junto con las coordenadas conocidas del dispositivo cliente en una fase fuera de línea. Esta información puede ser determinista [1] o probabilística. [3] Durante la fase de seguimiento en línea, el vector RSSI actual en una ubicación desconocida se compara con los almacenados en la huella dactilar y la coincidencia más cercana se devuelve como la ubicación estimada del usuario. Estos sistemas pueden proporcionar una precisión media de 0,6 my una precisión de la cola de 1,3 m. [8] [10]

Su principal desventaja es que cualquier cambio en el entorno, como agregar o quitar muebles o edificios, puede cambiar la "huella digital" que corresponde a cada ubicación, lo que requiere una actualización de la base de datos de huellas digitales. Sin embargo, la integración con otros sensores, como la cámara, se puede utilizar para hacer frente a los cambios en el entorno.

Campo magnético, Barómetro, 
\subsection{Sistemas Inerciales}